\documentclass{article}
\bibliographystyle{plain}
%\usepackage[T1]{fontenc}
%\usepackage[utf8]{inputenc}
\usepackage{hyperref}
\usepackage{xcolor}
\usepackage[margin=0.8in]{geometry}
\usepackage{sansmath}

% paragraph separation
\setlength{\parskip}{0.5em}
% no indented paras 
\setlength{\parindent}{0pt}
% reduce gaps between sections

\newcommand{\HRule}{\rule{\linewidth}{0.5mm}}
\newcommand{\Hrule}{\rule{\linewidth}{0.3mm}}
\newcommand{\code}[1]{\mathsf{#1}}
\newcommand{\TODO}[1]{\textcolor{red}{#1}}
\newcommand{\aftersep}{\vspace{-0.3cm}}
\newcommand{\beforesep}{\vspace{-0.2cm}}

\makeatletter% since there's an at-sign (@) in the command name
\renewcommand{\@maketitle}{%
	\parindent=0pt% don't indent paragraphs in the title block
	\centering
	{\Large \bfseries\textsc{\@title}}
	\Hrule\par%
	\large{\textbf{\@author \hfill \@date}}
	\par
    \vspace{0.5cm}
}
\makeatother% resets the meaning of the at-sign (@)

\title{Research Statement}
\author{Annamira O'Toole}
\date{Ph.D. Applicant}

\begin{document}
\maketitle% prints the title block
%
Cryptography is a crucial tool for protecting an individual’s right to think and speak independently. My research focuses on authenticated private information retrieval schemes, a core tool for building private web browsing and communications when applied to problems like key transparency, certificate auditing, and key retrieval. In the past, I have also worked in zero-knowledge theory and implementation. \\

\aftersep
\subsection*{Educational Background}
\beforesep

I am currently concluding my MSc in computer science jointly between ETH Zurich and EPFL in Switzerland, thanks to funding from the \textbf{Master Excellence Fellowship program} \footnote{\href{https://www.epfl.ch/education/master/master-excellence-fellowships/}{Merit-based scholarship awarded by EPFL covering tutition and living expenses.}}. In Zurich, I am working as a \textbf{research assistant in Professor Kenneth Paterson’s Applied Cryptography Group}. In early 2025, I will complete my \textbf{MSc research thesis on anonymous credentials at IBM Research} in Zurich where I will be collaborating with Dr. Kaoutar El Khiyaoui and Dr. Elli Androulaki. In 2023, I graduated from UC Berkeley with degrees in mathematics and computer science, where I also gained a broad exposure to policy through economics and law. My industry experience spans investment-focused engineering at a hedge fund \footnote{\href{www.bridgewater.com/}{Investment Logic Engineering internship at Bridgewater Associates.}}, analyzing interest rates on decentralized token exchanges \footnote{\href{https://www.gauntlet.xyz/}{Semester-long engineering internship at Gauntlet Network, an R\&D firm in the blockchain space.}}, and building randomized PCA for genomics datasets \footnote{\href{https://hail.is/}{A contribution to the Python library called Hail, maintained by a team at the Broad Institute of MIT and Harvard}}. I have previously contributed to the cryptography package $\code{arkworks.rs}$ \footnote{\href{https://github.com/arkworks-rs}{A contribution to a system for elastic proofs within the Rust $\code{arkworks.rs}$ package.}}, while exploring polynomial commitment schemes for space-efficient SNARKs, essential for modern anonymous credential verification systems and verifiable distributed computation. 

\aftersep
\subsection*{Personal Motivation}
\beforesep

I view my research as inseparable from the world it is deployed in. I realized the impact of cryptography while learning about how the NSA hid a backdoor in NIST’s standardization of the $\code{Dual\_EC\_DRBG}$ elliptic curve pseudorandom number generator, leaked during the 2013 Snowden revelations \TODO{CITE}. Curious, I watched the documentary \textit{Citizenfour} \TODO{CITE}, which documents Snowden’s beliefs about whistleblowing—a process heavily reliant on privacy tools like Tails OS \footnote{\href{https://tails.net/}{A privacy-preserving operating system that is a part of the global non-profit Tor Project.}}. The need for privacy-preserving systems grows as data and digital assets increase in value. However, if deploying security features comes at too high a cost, they will not be implemented in the real world. Unfortunately, the resource-efficiency of cryptography comes with a direct tradeoff against the privacy it provides. Furthermore, as most of today’s communications systems are controlled and standardized by governments, writing effective and enforceable privacy regulation is critical. 

\aftersep
\subsection*{Current Research}
\beforesep

My current work on private information retrieval (PIR) approaches an almost-ideal research question: it solves useful problems, has a beautiful theory, and yet suffers from shortcomings in implementation, which my work aims to solve. In Zurich, I experienced the \textbf{full research life cycle} from ideation, to protocol design trial-and-error, to writing security proofs, and implementation. My work involved surveying PIR literature and identifying the shortcomings of existing schemes, while closely collaborating with post-doc Dr. Francesca Falzon and PhD candidate Laura Hetz. Our project’s scheme builds the first \textbf{two-server authenticated PIR scheme} that will provide sublinear communication and computation, with a tradeoff on client storage. \textbf{We plan to submit to a conference in early 2025.} I now understand that applied cryptography systems often face drawbacks often cleverly-hidden by publications. For example, certain schemes only work efficiently for limited data formats, or require unusually large parameters to achieve the desired privacy guarantee -- drawbacks only ascertainable by digging into the weeds of highly technical privacy proofs. However, \textbf{it is exactly this investigative process that I learned to love, and that gives rise to open research questions with novel solutions.}

\aftersep
\subsection*{PhD Vision}
\beforesep

At the start of my PhD, I plan to \textbf{focus on building applied systems for well-defined use-cases, leaning on my algorithm design and engineering skills while building maturity within cryptography.} I dream of collaborating directly with privacy-preserving projects such as Signal (end-to-end encrypted messaging), Brave (private browsing), or Tor network (anonymous communications) to build protocols that are provably secure and well-designed for their immediate use-cases. Specifically, \TODO{[Target Advisor]’s} work on \TODO{[XX]} interests me as \TODO{[XX]}. As I build maturity during my PhD while working on applied systems, I hope to shift my work towards pure cryptographic contributions, such as signature schemes and authentication primitives. These are also active research areas of [Target Supervisor], such as \TODO{[XX]}. Finally, I want to understand on a deep level how cryptographers capture the notion of privacy mathematically. I am interested in reconciling inconsistencies between different notions of privacy that must be combined to prove the security of highly complex systems.

I intend the arc of my PhD to take me through several sub-fields of cryptography from \textbf{(1) building end-to-end applied systems} for specific privacy use-cases (which I am already working on), to \textbf{(2) improving upon widely-applicable cryptographic primitives}, to \textbf{(3) a theoretical contribution to cryptography.}

\aftersep
\subsection*{Interest in Data Privacy Regulation}
\beforesep

My interest in security research goes beyond answering technical questions. I am interested in exploring cases of government-compelled decryption, Section 230 corporate immunity, censorship of ill-defined disinformation, government immunity during classified surveillance operations, and the regulation of digital currencies as securities. These are also active areas of research at \TODO{[XXX University]} in \TODO{[XXX’s group]} and alongside my technical research, I hope to learn from and contribute to their work during my PhD.

\aftersep
\subsection*{Teaching \& Values}
\beforesep

In Zurich, I am a member of a vibrant international lab community -- from climbing mountains together to leading an internal study group for learning Rust programming. I have had moments of intense intellectual joy while working on my current research, and have formed close working relationships and friendships with my research mentors. I enjoy the process of struggling to keep up with another smart person, only to later slow down and understand their thinking step-by-step, and then offer my thoughts and ideas in return. Cultivating an ethical research environment and healthy learning environment for junior students is extremely important to me. To do so, I have enthusiastically been involved in teaching during my bachelors for UC Berkeley’s \textbf{undergraduate algorithms course CS 170} \TODO{CITE}, as well as with high school students in Jamaica by teaching and organizing \textbf{JamCoders} \footnote{\href{https://jamcoders.org.jm/}{A summer program in Kingston, Jamaica that teaches intro computer science to 50 high school students each summer.}}, and its sister program AddisCoders\footnote{\href{https://www.addiscoder.com/}{JamCoders' sister program in Addis Ababa, Ethiopia.}}. My teaching at UC Berkeley awarded me an \textbf{outstanding undergraduate instructor award} \footnote{\href{https://gsi.berkeley.edu/programs-services/award-programs/ogsi}{Annual awards to graduate and undergraduate teaching assistants nominated by UC Berkeley faculty and students.}}, and I am excited to continue teaching as a PhD student. 

\textbf{I believe in a future where large corporations as well as governments cannot easily use an individual's data to manipulate their behaviors}. This future requires advancements in anti-censorship systems, encrypted and anonymous communication, post-quantum cryptography, updated privacy-preserving internet protocols, and a legal framework for protecting an individual's privacy that is built in collaboration with computer security experts. I hope to help build such a future during and after my PhD. Thank you for your consideration.


% Regarding references: The bibtex file (applications.bib) contains references in an \textit{ad-hoc} format. You can add references and adapt as needed. For example, this is a reference for work in preparation \cite{Doe:inprep}. You can highlight your name with bold font to indicate first author publications \cite{Doe:2022}. This template was used for an application as a postdoctoral researcher in astronomy and astrophysics. You can find a list of journal macros in the preamble. Comment, adapt, and delete as needed.

% \cite{bridgewater}
% \bibliography{bib}

\end{document}

